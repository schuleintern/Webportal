Want to contribute to sabre/dav? Here are some guidelines to ensure your patch gets accepted.

\subsection*{Building a new feature? Contact us first }

We may not want to accept every feature that comes our way. Sometimes features are out of scope for our projects.

We don\textquotesingle{}t want to waste your time, so by having a quick chat with us first, you may find out quickly if the feature makes sense to us, and we can give some tips on how to best build the feature.

If we don\textquotesingle{}t accept the feature, it could be for a number of reasons. For instance, we\textquotesingle{}ve rejected features in the past because we felt uncomfortable assuming responsibility for maintaining the feature.

In those cases, it\textquotesingle{}s often possible to keep the feature separate from the sabre projects. sabre/dav for instance has a plugin system, and there\textquotesingle{}s no reason the feature can\textquotesingle{}t live in a project you own.

In that case, definitely let us know about your plugin as well, so we can feature it on \href{http://sabre.io/}{\tt sabre.\+io}.

We are often on \href{irc://freenode.net/#sabredav}{\tt I\+RC}, in the \#sabredav channel on freenode. If there\textquotesingle{}s no one there, post a message on the \href{http://groups.google.com/group/sabredav-discuss}{\tt mailing list}.

\subsection*{Coding standards }

sabre projects follow\+:


\begin{DoxyEnumerate}
\item \href{http://www.php-fig.org/psr/psr-1/}{\tt P\+S\+R-\/1}
\item \href{http://www.php-fig.org/psr/psr-4/}{\tt P\+S\+R-\/4}
\end{DoxyEnumerate}

sabre projects don\textquotesingle{}t follow \href{http://www.php-fig.org/psr/psr-2/}{\tt P\+S\+R-\/2}.

In addition to that, here\textquotesingle{}s a list of basic rules\+:


\begin{DoxyEnumerate}
\item P\+HP 5.\+4 array syntax must be used every where. This means you use {\ttfamily \mbox{[}} and {\ttfamily \mbox{]}} instead of {\ttfamily array(} and {\ttfamily )}.
\item Use P\+HP namespaces everywhere.
\item Use 4 spaces for indentation.
\item Try to keep your lines under 80 characters. This is not a hard rule, as there are many places in the source where it felt more sensibile to not do so. In particular, function declarations are never split over multiple lines.
\item Opening braces ({\ttfamily \{}) are {\itshape always} on the same line as the {\ttfamily class}, {\ttfamily if}, {\ttfamily function}, etc. they belong to.
\item {\ttfamily public} must be omitted from method declarations. It must also be omitted for static properties.
\item All files should use unix-\/line endings ({\ttfamily \textbackslash{}n}).
\item Files must omit the closing php tag ({\ttfamily ?$>$}).
\item {\ttfamily true}, {\ttfamily false} and {\ttfamily null} are always lower-\/case.
\item Constants are always upper-\/case.
\item Any of the rules stated before may be broken where this is the pragmatic thing to do.
\end{DoxyEnumerate}

\subsection*{Unit test requirements }

Any new feature or change requires unittests. We use \href{http://phpunit.de/}{\tt P\+H\+P\+Unit} for all our tests.

Adding unittests will greatly increase the likelyhood of us quickly accepting your pull request. If unittests are not included though for whatever reason, we\textquotesingle{}d still {\itshape love} your pull request.

We may have to write the tests ourselves, which can increase the time it takes to accept the patch, but we\textquotesingle{}d still really like your contribution!

To run the testsuite jump into the directory {\ttfamily cd tests} and trigger {\ttfamily phpunit}. Make sure you did a {\ttfamily composer install} beforehand. 